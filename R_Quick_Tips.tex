\documentclass{beamer}
\usepackage[utf8]{inputenc}
\usepackage{verbatim}
\usepackage{tikz}
\usepackage{hyperref}
\setbeamertemplate{footline}[frame number]
\title{Introduction to R}
\author{Heide Jackson }
\date{August 2019}

\begin{document}

\maketitle

\begin{frame}{Introductions}
\begin{itemize}
\item Name?
\item Affiliation?
\item Experience using other statistical processing programs (SAS?, R? SPSS? Stata?)
\item Something you'd like to get out of this class?
\end{itemize}
\end{frame}

\begin{frame}{Objectives for Today}
\begin{itemize}
    \item Know how to install R and RStudio.
    \item Understand the concept of packages and how to find out more about them.
    \item Summarize a dataset and variables within it.
    \item Manipulate a variable and adding a new variable to a dataset.
    \item Perform linear regression.
    \item Save R sessions and data files.
\end{itemize}
\end{frame}

\begin{frame}{Opening R}
\begin{itemize}
    \item In today's class we are going to discuss the basics of using R.
    \item In this we are going to be using R as accessed through R Studio (V1.2.1335).

\end{itemize}
\end{frame}

\begin{frame}{Downloading and Installing R}
\begin{itemize}
    \item If you are working from a computer lab computer, you should be able to directly connect to R Studio.
    \item If you are working from your personal computer, you will need to download R and R Studio.  Here's how:
    \begin{itemize}
        \item Download R from \url{https://cran.rstudio.com}
    \item Next, you'll want to install RStudio from \url{https://www.rstudio.com/}
   \end{itemize} 
\end{itemize}
\end{frame}

\begin{frame}{Opening R Studio}
\begin{tikzpicture}
\node [anchor=south west,inner sep=0] (image) at (0,0) {\includegraphics[width=.9\textwidth]{Rstudio.PNG}};
\end{tikzpicture}
\end{frame}

\begin{frame}{Starting an R Script}
\begin{itemize}
    \item We are going to start by creating a project directory that will house the R scripts, graphs, and data created during this session.
    \item To do this, go to file, new project, and create a new directory titled Intro_to_R
\end{itemize}
\end{frame}

\end{document}
