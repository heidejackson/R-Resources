\documentclass{beamer}
\usepackage[utf8]{inputenc}
\usepackage{verbatim}
\usepackage{tikz}
\usepackage{hyperref}
\setbeamertemplate{footline}[frame number]
\title{Simulating Data in R}
\author{\texorpdfstring{Heide Jackson \newline\url{heidej@umd.edu}}{Author}}
\institute{University of Maryland Population Research Center}

\date{August 2019}

\begin{document}
\maketitle
\begin{frame}{High Level Things to Know}
\begin{itemize}
\item R makes creating "fake" data very easy.
\item Setting a seed makes findings reproducible
\item Simulated data allows us to test model inferences and the sensitivity of results to violations of our assumptions. 
\end{itemize}
\end{frame}


\begin{frame}[fragile]{Simulating Different Types of Distributions}
\begin{itemize}
    \item Different functions can easily generate different types of variables.
\end{itemize}
\begin{verbatim}
#series of integers
id<-seq(1:100) 
# 100 cases, mean 0, sd 1 normally distributed
error<-rnorm(n=100, mean=0, sd=1) 
#binomial sample, values 0 and 1, probability of 1, .3
missing<-rbinom(n=100,1,.3)
#simulate a four category var, different probabilities
x3<-sample(x=0:3, size=100, replace=TRUE, 
prob=c(.2,.1,.4,.3))
\end{verbatim}
\end{frame}


\begin{frame}[fragile]{Simulating Relationships}
\begin{itemize}
    \item Objects can also be created relative to other simulated data.
\end{itemize}
\begin{verbatim}
y2<-.5*x1+-.1*x2+error+.3*missing
y=ifelse(missing==1,NA,y2)
\end{verbatim}
\item For analysis, it is helpful to wrap this up into a data frame.
\begin{verbatim}
fakedata<- data.frame(y2, x1, x2, missing)
\end{verbatim}
\end{frame}

\begin{frame}{Running Simulated Data within a Function}
\begin{itemize}
    \item Simulating data within a broader function can be helpful for power calculations and modifying distributional assumptions.
\end{itemize}
\end{frame}
\end{document}
