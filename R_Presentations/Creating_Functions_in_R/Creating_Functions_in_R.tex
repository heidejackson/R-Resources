\documentclass{beamer}
\usepackage[utf8]{inputenc}
\usepackage{verbatim}
\usepackage{tikz}
\usepackage{amsmath}
\usepackage{listings}
\usepackage{hyperref}

\setbeamertemplate{footline}[frame number]
\title{Creating R Functions}
\author{\texorpdfstring{Heide Jackson \newline\url{heidej@umd.edu}}{Author}}
\institute{University of Maryland Population Research Center}

\date{August 2019}

\begin{document}
\maketitle
\begin{frame}{High Level Things to Know}
\begin{itemize}
\item Custom functions can streamline R coding and group together related tasks.
\item Functions are available in base R, via packages added, and can be readily created.
\end{itemize}
\end{frame}


\begin{frame}[fragile]{The Structure of a Function}
\begin{itemize}
    \item All functions can be defined with a common structure
\end{itemize}
\begin{verbatim}
#function name
myfirstfunction<-
#function denotes we are defining a function
function
#parantheses contain objects defined by function
(){
#within brackets give what the function does
print("Hello")
}
\end{verbatim}
\end{frame}


\begin{frame}[fragile]{Viewing and Calling the Function}
\begin{itemize}
    \item To view what is in the user created function, we can just type myfirstfunction.
    \item Running the function is similar type myfirstfunction()
\end{itemize}
\end{frame}

\begin{frame}{Useful Functions within a Function}
\begin{itemize}
    \item paste()--combines text or objects together.
    \item assign()--assigns function, text, or something else to an object.
    \item if(), else(), and print--if/else statements in combination with print can show warning messages or document function functioning.
\end{itemize}
\end{frame}
\begin{frame}[fragile]{An Example of a Function in Action}{Adding a Prefix in a Data Frame}
\lstset{language=R} 
\begin{lstlisting}

addprefix<- function(data,prefix){
  tnames<-colnames(data)
  nnames<-paste(prefix,tnames,sep="")
  colnames(data)<-nnames
  print(paste("All Variables Renamed with Prefix", prefix,
  sep=" "))
  return(data)
}
library("gapminder")
addprefix(gapminder, "n")
\end{lstlisting}
\end{frame}
\begin{frame}{Other Useful Things to Know}
    \begin{itemize}
        \item Objects defined for a function exist only in the context of that function.
        \item When calling a function, R matches objects to function names based on position.
        \item Source function can be used to load in functions stored in other scripts.
    \end{itemize}
\end{frame}
\end{document}
